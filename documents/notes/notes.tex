\documentclass{article}

\usepackage[ngerman]{babel}
\usepackage[utf8]{inputenc}
\usepackage[T1]{fontenc}
\usepackage{hyperref}
\usepackage{csquotes}

\usepackage[
    backend=biber,
    style=apa,
    sortlocale=de_DE,
    natbib=true,
    url=false,
    doi=false,
    sortcites=true,
    sorting=nyt,
    isbn=false,
    hyperref=true,
    backref=false,
    giveninits=false,
    eprint=false]{biblatex}
\addbibresource{../references/bibliography.bib}

\title{Notizen zum Projekt Data Ethics}
\author{Name des Autors}
\date{\today}

\begin{document}
\maketitle

\abstract{
    Dieses Dokument ist eine Sammlung von Notizen zu dem Projekt. Die Struktur innerhalb des
    Projektes ist gleich ausgelegt wie in der Hauptarbeit, somit kann hier einfach geschrieben
    werden, und die Teile die man verwenden möchte, kann man direkt in die Hauptdatei ziehen.
}

\tableofcontents

/section{Einleitung}

Künstliche Intelligenz (KI), auch bekannt als artifizielle Intelligenz, ist ein Zweig der Informatik, der sich mit der Automatisierung intelligenter Verhaltensweisen sowie dem maschinellen Lernen beschäftigt. Sie wird oft als "zukunftsweisende Technologie" bezeichnet. KI kann auch menschliche Fähigkeiten, wie beispielsweise Kreativität, nachahmen.

Neben den vielen positiven Aspekten verbinden wir Menschen jedoch auch große Furcht mit der Künstlichen Intelligenz, insbesondere hinsichtlich der möglichen Folgen. Es beunruhigt uns, dass die Gefahr besteht, in der Berufswelt von KI ersetzt zu werden. Diese Sorgen sind allerdings größtenteils unbegründet.

Dennoch stellt sich für uns die Frage, wie KIs überhaupt funktionieren. KIs haben das Ziel, die Art und Weise, wie menschliches Denken und Lernen funktioniert, auf einen Computer zu übertragen. Durch KIs lernt der Computer eigenständig Entscheidungen zu treffen, Antworten zu finden und Probleme zu lösen. Um diesem Ziel näher zu kommen, ist ein umfassendes und äußerst komplexes Training erforderlich. Der Software wird beigebracht, in welcher Situation welche Entscheidungen zu treffen sind. Dazu werden Trainingsdaten gebraucht. Unter Trainingsdaten wird eine Reihe von Informationen vestanden. Diese werden den Maschinen zur Verfügung gestellt, um sie zu lehren und auszubilden. Dies kann man sich an dem Beispiel der Unterscheidung verschiedener Religionen, nämlich beispielsweise dem Judentum und Christentum vorstellen. Es wird ihnen ein Computeralgorithmus mit Beispielen für jeden dieser Algorithmen bereitgestellt, dadurch erlernen sie im Laufe der Zeit weitere Unterscheidungsmerkmale. Es können sich um Unterschiede handeln, wie beispielsweise, dass die Juden sich als auserwähltes Volk sehen, wärhend die Christen glauben, die Erben und Nachfahren des Volkes Isreal zu sein. Aus dieser Art, wie KI trainiert wird, kann sich auch die Gefahr ergeben, dass, da die KI eine  Abbildung der Daten lernt, dadurch auch falsch trainiert werden könnte. Wenn nämlich die KI in christlichen Ländern, wie der Schweiz durch fast ausschliesslich Trainingsdaten in Form von Fakten über das Christentum trainiert wird, kann die Gefahr bestehen, dass den Fakten über das Judentum nicht die gleiche Wichtigkeit begemeissen wird, wie den Daten über das Christentum. 
Nach abgeschlossenem Training besitzt die KI die Fähigkeit, mit Hilfe der Trainingsdaten und vom Benutzer bereitgestellten Eingabedaten selbstständig Entscheidungen zu treffen.



\begin{itemize}
    \item Wie gut wird der Zusammenhang von Daten und dem Training von KIs erklärt?
    \item Macht die Einleitung klar wie KIs funktionieren?
    \item Gibt es eine gute Überleitung zur Fragestellung?

\end{itemize}

/section{Fragestellung}

In welchen Bereichen und auf welche Weise beeinflusst der Einsatz von KI das Leben und die Entwicklung von Kindern?

-gute Dinge
-ethische Probleme durch KI 

\input{section_ai.tex}

\printbibliography

\end{document}
