\documentclass{article}

\usepackage[ngerman]{babel}
\usepackage[utf8]{inputenc}
\usepackage[T1]{fontenc}
\usepackage{hyperref}
\usepackage{csquotes}

\usepackage[
    backend=biber,
    style=apa,
    sortlocale=de_DE,
    natbib=true,
    url=false,
    doi=false,
    sortcites=true,
    sorting=nyt,
    isbn=false,
    hyperref=true,
    backref=false,
    giveninits=false,
    eprint=false]{biblatex}
\addbibresource{../references/bibliography.bib}

\title{Notizen zum Projekt Data Ethics}
\author{Name des Autors}
\date{\today}

\begin{document}
\maketitle

\abstract{
    Dieses Dokument ist eine Sammlung von Notizen zu dem Projekt. Die Struktur innerhalb des
    Projektes ist gleich ausgelegt wie in der Hauptarbeit, somit kann hier einfach geschrieben
    werden, und die Teile die man verwenden möchte, kann man direkt in die Hauptdatei ziehen.
}

\tableofcontents

/section{Einleitung}

Künstliche Intelligenz (KI), auch bekannt als artifizielle Intelligenz, ist ein Zweig der Informatik. Sie beschäftigt sich mit der Automatisierung intelligenter Verhaltensweisen und dem maschinellen Lernen. Die KI kann ausserdem menschliche Fähigkeiten, wie beispielsweise Kreativität nachahmen. 
Wir Menschen jedoch verbinden mit der Künstlichen Intelligenz einerseits viele positive Aspekte, doch wir fürchten uns auch stark von den damit verbundenen Folgen. Es beunruhigt uns, dass die Gefahr bestehen könnte, dass wir beispielsweise in der Berufswelt von der KI ersetzt werden könnten. Die Sorgen, die wir diesbezüglich haben sind allerdings mehrheitlich überflüssig. 
• Wie gut wird der Zusammenhang von Daten und dem Training von KIs erklärt?
• Macht die Einleitung klar wie KIs funktionieren?
• Gibt es eine gute Überleitung zur Fragestellung?

/section{Fragestellung}

In welchen Bereichen und auf welche Weise beeinflusst der Einsatz von KI das Leben und die Entwicklung von Kindern?



\input{section_ai.tex}

\printbibliography

\end{document}
