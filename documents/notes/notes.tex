\documentclass{article}

\usepackage[ngerman]{babel}
\usepackage[utf8]{inputenc}
\usepackage[T1]{fontenc}
\usepackage{hyperref}
\usepackage{csquotes}

\usepackage[
    backend=biber,
    style=apa,
    sortlocale=de_DE,
    natbib=true,
    url=false,
    doi=false,
    sortcites=true,
    sorting=nyt,
    isbn=false,
    hyperref=true,
    backref=false,
    giveninits=false,
    eprint=false]{biblatex}
\addbibresource{../references/bibliography.bib}

\title{Notizen zum Projekt Data Ethics}
\author{Nora Liechty}
\date{\today}

\begin{document}
\maketitle

\abstract{
    Dieses Dokument enthält alle Notizen zu dem Projekt Data Ethics.
}

\tableofcontents

\section{Einleitung}

Künstliche Intelligenz (KI), auch bekannt als artifizielle Intelligenz, ist ein Zweig der Informatik, der sich mit der Automatisierung intelligenter Verhaltensweisen sowie dem maschinellen Lernen beschäftigt. Sie wird oft als "zukunftsweisende Technologie" bezeichnet. KI kann auch menschliche Fähigkeiten, wie beispielsweise Kreativität, nachahmen.

Neben den vielen positiven Aspekten verbinden wir Menschen jedoch auch grosse Furcht mit der Künstlichen Intelligenz, insbesondere hinsichtlich der möglichen Folgen. Es beunruhigt uns, dass die Gefahr besteht, in der Berufswelt von KI ersetzt zu werden. Diese Sorgen sind allerdings grösstenteils unbegründet.
Dennoch stellt sich für uns die Frage, wie KIs überhaupt funktionieren. 
 
\section{Funktion}

KI funktioniert durch das Trainieren von Algorithmen mit grossen Datenmengen, um Muster zu erkennen und Vorhersagen zu treffen. Durch ständige Anpassung und Validierung wird die Genauigkeit und Leistungsfähigkeit des Systems verbessert. KIs haben das Ziel, die Art und Weise, wie menschliches Denken und Lernen funktioniert, auf einen Computer zu übertragen. Durch KIs lernt der Computer eigenständig Entscheidungen zu treffen, Antworten zu finden und Probleme zu lösen. Um diesem Ziel näherzukommen, ist ein umfassendes und äusserst komplexes Training erforderlich.

\section{Wie wird KI trainiert?}

Um der Software beizubringen, in welcher Situation welche Entscheidungen zu treffen sind, werden Trainingsdaten gebraucht. Unter Trainingsdaten wird eine Reihe von Informationen verstanden. Diese werden den Maschinen zur Verfügung gestellt, um sie zu lehren und auszubilden. 
Das Training der KI lässt sich in drei Phasen unterteilen. Im ersten Schritt werden grosse Mengen an Trainingsdaten eingespeist, um das System zu verbessern. Überwachtes Lernen nutzt gelabelte Daten, während unüberwachtes Lernen eigenständig Muster in nicht gelabelten Daten erkennt.
Im nächsten Schritt, der Validierung, wird das Modell mit neuen Daten geprüft und stoppt, wenn keine Verbesserungen mehr möglich sind. Im letzten Schritt, der sogenannten Testphase, wird die KI mit realen Daten geprüft, um ihre praktische Einsatzfähigkeit und Anpassungsfähigkeit sicherzustellen.

Aus dieser Art, wie KI trainiert wird, kann sich auch die Gefahr ergeben, dass, da die KI eine Abbildung der Daten lernt, dadurch auch falsch trainiert werden könnte. Es kann auch zu ethischen Problemen kommen.
Wenn eine KI nämlich immer verzerrte oder einseitige Daten erhält, übernimmt sie diese Vorurteile, was zu diskriminierenden Entscheidungen führt. Beispielsweise, wenn die KI immer nur hört, dass dunkelhäutige Menschen kriminell sind, wird sie diese schädlichen Stereotypen verstärken. Dies beeinflusst die öffentliche Wahrnehmung und festigt Vorurteile in der Gesellschaft.

Nach abgeschlossenem Training besitzt die KI die Fähigkeit, mithilfe der Trainingsdaten und vom Benutzer bereitgestellten Eingabedaten selbstständig Entscheidungen zu treffen. Von dieser Fähigkeit können Personen aller Altersgruppen profitieren. Was hat die KI aber im Genauen für einen Einfluss auf das Leben und die Entwicklung von Kindern.

\section{Fragestellung}

Da mich die Auswirkung der Nutzung von KI insbesondere auf Kinder stark interessiert, stellt sich für mich die Frage, in welchen Bereichen und auf welche Weise der Einsatz von KI das Leben und die Entwicklung von Kindern beeinflusst. Diese Fragestellung steht somit im Zentrum meiner Arbeit über KI und Ethik.

\section{KI und Kinder}

Kinder heutiger Generationen kommen schon im jungen Alter mit künstlicher Intelligenz in Kontakt, ohne sich dem überhaupt bewusst zu sein. Sie ist heutzutage nämlich in vielen speziell für Kinder entwickelten Produkten und Dienstleistungen integriert.
Ein typisches Beispiel sind intelligente Spielzeuge, die auf Sprache und Bewegungen reagieren. Diese Spielzeuge können von den Kindern lernen, sich deren Verhalten und Vorlieben anzupassen und dadurch ein personalisiertes Spielerlebnis zu bieten.
Auch im Bildungsbereich findet KI immer häufiger Anwendung. Interaktive Lern-Apps passen sich dem Lernfortschritt der Kinder an und ermöglichen so eine individuelle Förderung. Zudem unterstützen KI-basierte Anwendungen Kinder dabei, kreative Fähigkeiten zu entwickeln, etwa beim Erlernen eines Musikinstruments.

Darüber hinaus kann künstliche Intelligenz einen wichtigen Beitrag zur Sicherheit der Kinder leisten. Es gibt Überwachungssysteme, die Eltern benachrichtigen, wenn ihr Kind das Haus verlässt. Diese KI-Anwendungen gestalten den Alltag sicherer und vermitteln Eltern ein Gefühl der Beruhigung. 

Neben all diesen Vorteilen darf nicht vergessen werden, dass Kinder ein ethisches Verständnis für diese Technologien entwickeln müssen. Sie sollten sich bewusst sein, dass KI-Systeme zwangsläufig so programmiert werden müssen, dass sie alle Menschen gleich und ohne Vorurteile behandeln. Es ist ausserdem wichtig, dass die Kinder verstehen, dass hinter jeder Maschine Entwickler und Unternehmer stehen, die für ihr Handeln aufkommen müssen.

Die Tatsache, dass KI nicht in der Lage ist, menschliches Einfühlungsvermögen und moralisches Urteilsvermögen zu ersetzen, sollte den Kindern ebenfalls bekannt sein. 

\section{Positive Aspekte}

\begin{itemize}

    \item Förderung einer positiven und verantwortungsvollen Haltung gegenüber Technologie
    \item Verbesserung der Medienkompetenzen
    \item Unterstützung bei kritischen Denkfähigkeiten
    \item Förderung der Sprachentwicklung durch interaktive und personalisierte Sprachlernprogramme
    \item Steigerung der Kompetenzen im Lern- und Schreibprozess

\end{itemize}

\section{Negative Aspekte}

\begin{itemize}

    \item Einfluss auf die Konzentrationsfähigkeit und das Sozialverhalten
    \item Potenzielle Abhängigkeit von Technologie und Verringerung der zwischenmenschlichen Interaktion
    \item Reduzierte Kreativität und Problemlösungsfähigkeiten
    \item Rückgang des kritischen Denkens
    \item Entstehung sozialer Ungleichheiten aufgrund unterschiedlichen Zuganges zu technologischen Ressourcen

\end{itemize}

\section{Fazit}

Der Einsatz von KI beeinflusst das Leben und die Entwicklung von Kindern in verschiedenen Bereichen und auf vielfältige Weise. In der Bildung unterstützt KI personalisierte Lernansätze und verbessert den Zugang zu Bildungsmaterialien. In der Gesundheitsversorgung ermöglicht KI bessere Diagnosen und den Wünschen entsprechende Behandlungspläne. In der Freizeit bietet KI interaktive Lern- und Unterhaltungsangebote. 
Jedoch ist eine sorgfältige Überwachung und Regulierung des Einsatzes von KI im Kinderleben erforderlich, um Datenschutz, Ethik und Sicherheit vollständig gewährleisten zu können. 

\input{section_ai.tex}

\printbibliography

\end{document}
