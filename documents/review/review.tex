\documentclass{article}

\usepackage[ngerman]{babel}
\usepackage[utf8]{inputenc}
\usepackage[T1]{fontenc}
\usepackage{hyperref}
\usepackage{csquotes}

\usepackage[
    backend=biber,
    style=apa,
    sortlocale=de_DE,
    natbib=true,
    url=false,
    doi=false,
    sortcites=true,
    sorting=nyt,
    isbn=false,
    hyperref=true,
    backref=false,
    giveninits=false,
    eprint=false]{biblatex}
\addbibresource{../references/bibliography.bib}

\title{Review des Papers "Ethik im Umgang mit Daten" von Mara Schöni}
\author{Nora Liechty}
\date{\today}

\begin{document}
\maketitle

\abstract{
    Dies ist ein Review der Arbeit zum Thema Ethik im Umgang mit Daten von Mara Schöni. 
}

\section{Positive Aspekte}

\begin{itemize}

    \item Klar strukturierte Einleitung mit verständlicher Definition, die die Anwendungsbeispiele sowie den Nutzen der KI verdeutlicht und dem Leser einen guten Einstieg in das Thema gibt. 
    \item Behandlung ethischer Überlegungen, welche in der Diskussion über die KI-Nutzung nicht vernachlässigt werden dürfen. 
    \item Ausführliche Beschreibung des Trainingsprozesses und Erklärung verschiedener Phasen
    \item Beleuchtung der positiven als auch der negativen Auswirkungen von KI auf die Gesellschaft
    \item Vermittlung eines ausgewogenen Bildes 
    
\end{itemize}

\section{Negative Aspekte}

\begin{itemize}

    \item Mangel an Zitaten, was dazu führen könnte, dass die Glaubwürdigkeit der dargestellten Informationen ein wenig beeinträchtigt wird. 
    \item Fehlende tiefergehende Analyse und kritische Auseinandersetzung, besonders bei Themen wie den ethischen Herausforderungen, hätten mehr Details und Beispiele hinzugefügt werden können. 
    \item Oberflächliche Darstellung des Abschnitts über den Einfluss von KI auf den IQ
    \item Sprachliche und formale Rechtschreib- und Grammatikfehler 
    
\end{itemize}

\section{Verbesserungsvorschläge}

\begin{itemize}

    \item Nutzung von Zitaten, um wichtige Aussagen zu untermauern und für Glaubwürdigkeit der Informationen zu sorgen.
    \item Tiefere Analyse ethischer Fragen und Ergänzung durch Statistiken
    \item Erweiterung des Abschnitts über die Auswirkungen von KI auf den Arbeitsmarkt 
    \item Integrierung von Fallbeispielen für bessere Veranschaulichung und Verdeutlichung des praktischen Nutzens sowie der Herausforderungen der KI
    \item Einbringung von Forschungsergebnissen in gewisse Abschnitte, insbesondere im Abschnitt über den Einfluss von KI auf den IQ
    \item Vermeiden von Rechtschreib- und Grammatikfehlern durch eine gründliche Korrektur und sprachliche Überarbeitung.
    \item Verbesserung des Stils für die Gestaltung eines flüssigeren und ansprechenderen Textes
    
\end{itemize}

\printbibliography

\end{document}
