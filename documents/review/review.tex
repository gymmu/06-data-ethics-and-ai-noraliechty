\documentclass{article}

\usepackage[ngerman]{babel}
\usepackage[utf8]{inputenc}
\usepackage[T1]{fontenc}
\usepackage{hyperref}
\usepackage{csquotes}

\usepackage[
    backend=biber,
    style=apa,
    sortlocale=de_DE,
    natbib=true,
    url=false,
    doi=false,
    sortcites=true,
    sorting=nyt,
    isbn=false,
    hyperref=true,
    backref=false,
    giveninits=false,
    eprint=false]{biblatex}
\addbibresource{../references/bibliography.bib}

\title{Review des Papers "Ethik im Umgang mit Daten" von \dots}
\author{Name des Autors}
\date{\today}

\begin{document}
\maketitle

\abstract{
    Dies ist ein Review der Arbeit zum Thema Ethik im Umgang mit Daten von <Name des Authors>.
}

\section{Positive Aspekte}

\begin{itemize}

    \item Klar strukturierte Einleitung mit verständlicher Definition, die die Anwendungsbeispiele sowie den Nutzen der KI verdeutlicht und dem Leser einen guten Einstieg in das Thema gibt. 
    \item Behandlung ethischer Überlegungen, welche in der Diskussion über die KI-Nutzung nicht vernachlässigt werden dürfen. 
    \item Ausführliche Beschreibung des Trainingsprozesses und Erklärung verschiedener Phasen
    \item Beleuchtung der positiven als auch der negativen Auswirkungen von KI auf die Gesellschaft
    \item Vermittlung eines ausgewogenen Bildes 
    
\end{itemize}

\printbibliography

\end{document}
